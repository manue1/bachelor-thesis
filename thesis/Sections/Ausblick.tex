 \cleardoublepage
\section{Ergebnisse und Ausblick}

%Ergebnisse

Im Rahmen dieser Diplomarbeit wurde ein Lichtschnittsensor auf Basis einer intelligenten Kamera entworfen und implementiert. 

Nach dem Aufbau des Systems wurden mehrere Verfahren zur Detektion der subpixelgenauen Position der Laserlinie implementiert und validiert, um die f�r die gegebene Versuchsumgebung geeignetsten Verfahren zu ermitteln.

Der auf dem Center-Of-Mass-Verfahren basierende Schwellwertalgorithmus (Thr) und der Gau�5 lieferten die besten Ergebnisse in Bezug auf die Genauigkeit: Bei der Untersuchung der Residuen der reproduzierten 3D-Punkte von einer Regressionsebene weist der Thr-Algorithmus mit einer Abweichung von maximal $0,234$\,mm die h�chste Genauigkeit auf.

Bei der Analyse der Varianz des gesch�tzten Maximums der Laserlinie weist der BR-Filter die geringste Varianz auf.
Die gemessenen Varianzen sind jedoch allesamt sehr gro� verglichen mit der theoretisch errechneten. Als Ursache hierf�r ist ein deutlich sichtbares Speckle-Muster ausgemacht worden. 

Die Lage der Laserebene wird mithilfe des Least-Squares-Verfahrens bestimmt. Zur schnellen und komfortablen Kalibrierung wurde im Rahmen dieser Arbeit ein treppenf�rmiger Kalibrierk�rper erstellt. Somit kann das System vor Ort mit dem gew�nschten Triangulationswinkel manuell angebracht werden und die exakte Lage der Ebene durch Aufnahme der auf den Kalibrierk�rper projizierten Laserlinie bestimmt werden.

Die Kamerakalibrierung wurde mit dem Kalibrierverfahren nach Zhang auf einem PC durchgef�hrt. Hierf�r wird das Bild via TCP/IP auf den Rechner �bertragen.

% Ausblick

F�r die Verarbeitung und Auswertung der erzeugten Tiefenbilder ist angedacht, ein herk�mmliches 2D-Bildverarbeitungs und -vermessungssystem zu verwenden. 

Zwei M�glichkeiten stehen mit der gegebenen Hardware zu Verf�gung: Entweder geschieht die Verarbeitung auf einem zus�tzlichen PC. Hierf�r kann das erzeugte Tiefenbild via TCP/IP �bermittelt werden. Im Rahmen dieser Arbeit entstand eine Klasse, die den Empfang auf PC-Seite erm�glicht. In diesem Fall w�rde die Kamera als reiner 3D-Sensor agieren. Durch die Netzwerkanbindung ist es m�glich, dass mehrere verteilte 3D-Sensoren Tiefenbilder produzieren, die alle auf einem Rechner ausgewertet werden k�nnen.

Der langfristig w�nschenswerte Zustand jedoch wird sicherlich die Verarbeitung und Vermessung des Bildes auf der Kamera sein, um somit unabh�ngig von herk�mmlichen PCs zu arbeiten, und getroffene Entscheidungen direkt �ber SPS-Schnittstellen an die Fertigungsstra�e zu geben.

Die erzielte Geschwindigkeit des in dieser Arbeit implementierten Systems ist von der Bildaufnahmefrequenz der eingesetzten Kamera beschr�nkt.
%Verglichen mit auf dem Markt erh�ltlichen Hochgeschwindigkeitslaserscannern die Detektionsraten von bis zu 35.000 Profilen pro Sekunde aufweisen ist das das Verfahren auf der eingesetzten intelligenten Kamera prinzipbedingt eher langsam, da die Aufnahmegeschwindigkeit durch die Aufnahmezeit der Kamera begrenzt ist. Eine sinnvolle Erweiterung w�re  Allerdings l�sst es sich kosteng�nstig realisieren.
Realistische Einsatzbereiche f�r ein solches System sind deswegen die Prototypgenerierung und stichprobenartige Qualit�tskontrollen, bei der die Objektverschiebung relativ zum Sensor zum Beispiel auf einem Drehteller durchgef�hrt wird. Eine m�gliche Erweiterung in Punkto Geschwindigkeitssteigerung w�re es, eine Kamera einzusetzen, die in der Lage ist, lediglich Teile des Bildaufnehmers auszulesen und zu �bertragen.




