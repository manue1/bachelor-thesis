\color{darkred}


\textbf{\Large{Erl�uterungen zum Latex-Template}}


\medskip
\medskip
\medskip

\textbf{Zur Verwendung}\index{Verwendung}

Das vorliegende Latex-Template sollte urspr�nglich vorrangig den Studenten der Informatik und der Ingenieurswissenschaften der Universit�t Karlsruhe (TH) als Vorlage f�r die Erstellung von Studien- oder Diplomarbeiten dienen. Nat�rlich steht es aber auch jedem anderen Studenten oder Promovenden zur Verf�gung, der Nutzen daraus ziehen kann.

\medskip
\medskip
\medskip


Das Dokument ist aufgebaut bzw. zu verwenden wie folgt:
\index{Template!Verwendung}\index{Template!Aufbau}
\begin{enumerate}\index{Deckblatt}
\item Das Deckblatt, das Format des Gesamtdokumentes, das Format der Literaturquellen usw. entspricht dem Stil und den Auf"-lagen wie sie bei uns am Lehrstuhl festgelegt sind.
\item Die Gliederung der Kapitel (Chapters) kann erfahrungsgem�� zu rund 50\,\% \dots 70\,\% �bernommen werden, muss aber nat�rlich entsprechend angepasst und in Sections und Subsections verfeinert werden. Somit ist das Rahmenwerk festgelegt.
\item Im F�lltext (lorem ipsum...) sind viele Beispiele zu fast allen relevanten Einbettungsobjekten eingef�gt: Tabellen, Formeln, Vektorgrafiken (CAD, UML-Diagramme...), Pixelgrafiken (Fotos, Screenshots), Charts, Algorithmen, \dots
\item Der erkl�rende Text zu den Einbettungsobjekten oder anderen Formatierungsmerkmalen ist in Kursivschrift formatiert.
\item Wenn statt des verwendeten Report-Styles der sog. Article-Style verwendet werden soll, so sind die Chapters durch Sections, die Sections durch Subsections usw. zu ersetzen. Wenn gew�nscht wird, die Kapitelanf�nge auf rechter Seite beizubehalten, so ist weiterhin folgende Anpassung notwendig: In der Datei Diplomarbeit.tex ist nach jedem \verb$\include$ ein \verb$\cleardoublepage$ einzuf�gen.
\end{enumerate}

\medskip
\medskip
\medskip


Es ergeben sich entsprechend zwei Nutzungsszenarien:\index{Nutzung}

\begin{enumerate}
\item Als Rahmenwerk. Hierf�r k�nnen Deckblatt, Hauptdokument und Kapitel-Dateien genutzt und mit eigenem Inhalt gef�llt werden, der urspr�ngliche Inhalt der Kapitel-Dateien wird einfach gel�scht bzw. �berschrieben.
\item Als Nachschlagewerk f�r bestimmte Formatierungen. Wenn zum Beispiel in der Diplomarbeit eine Grafik eingef�gt werden soll, so kann der Anwender im .pdf des vorliegenden Dokumentes eine �hnliche Grafik suchen, den Kommentar (in Kursivschrift) hierzu studieren und den entsprechenden zu Grunde liegenden Latex-Quelltext vergleichen.

\end{enumerate}

\clearpage
\color{darkred}



\textbf{Lizenz}\index{Lizenz}\index{Template!Lizenz}

Das Template darf angepasst, ver�ndert, erweitert und auch kommerziell vertrieben werden. Die einzige Auf\/lage ist, dass die Quelle des Templates in den Literaturquellen genannt und im Text als Quelle referenziert wird. Hierzu ist dem Text ein kurzer Satz beizuf�gen, und am Ende ist die Quelle einzuf�gen:

\begin{itemize}
\item Einzuf�gende Textzeile (Fu�note):

Der vorliegende Text ist auf Basis des Latex-Templates zu [1] erstellt.

\item Einzuf�gende zugeh�rige Quelle:

[1] T. \mbox{Gockel}. Form der wissenschaftlichen Ausarbeitung. Springer-Verlag, Heidelberg, 2008. Begleitende Materialien unter \url{http://www.formbuch.de}.

\end{itemize}

Weiterhin ist es sinnvoll, bei der Weitergabe des Templates die Latex-Quellen und die PDF-Datei nicht zu trennen.


\medskip
\medskip


\textbf{Danksagung}\index{Danksagung}

An Beispielen im F�lltext enth�lt der vorliegende Text Ausz�ge aus den Arbeiten der Kollegen Pedram Azad, Andreas B�ttinger, Alexander Bierbaum und Joachim Schr�der. Vielen Dank f�r die Bereitstellung dieser Ausz�ge.

Karlsruhe, den \today

Tilo Gockel


\vspace{1cm}

Kontakt:\\
\verb$info@formbuch.de$\\

Website:\\
\url{http://www.formbuch.de}\\


\vspace{2cm}

Hinweis

Die Informationen in diesem Dokument werden ohne R�cksicht auf einen eventuellen Patentschutz ver�ffentlicht. Die erw�hnten Soft- und Hardware-Bezeichnungen k�nnen auch dann eingetragene Warenzeichen sein, wenn darauf nicht besonders hingewiesen wird. Sie geh�ren den jeweiligen Warenzeicheninhabern und unterliegen gesetzlichen Bestimmungen. Verwendet werden u.\,a. folgende gesch�tzte Bezeichnungen: ActivePerl, Copernic Desktop Search, Google, Wikipedia, Microsoft Word, Office, Excel, Windows, Project, Adobe Acrobat, Adobe Reader, Adobe Photoshop, CorelDRAW, Corel PhotoPaint, Corel Paint Shop Pro, \mbox{TeXaide}.

\color{black}
