\chapter{Format der Parameterdateien}\index{Parameter}
\label{anhang_b}

\textcolor{darkred}{Anmerkung: Ein Anhang zur Erkl�rung der zum System zugeh�rigen Parameterdateien hat sich als sinnvoll und hilfreich erwiesen, damit nach Abschied des Diplomanden auch uneingeweihte Personen ohne Quelltextsichtung das System zumindest f�r Demozwecke in Betrieb nehmen k�nnen.}

\medskip


Die Parameterdateien enthalten die Information �ber die Punktmuster, bzw. Plattenstapel, welche im Rahmen der Testfeldkalibrierung f�r Kamera und Projektor verwendet werden. Es sind dies: Anordnung und Anzahl der Punkte auf dem Testfeldmuster und Plattendicke und -anzahl.\\\\
%\vspace{33pt}\\
\textbf{Kamerakalibrierung}, Datei \verb$world_camera.txt$\vspace{11pt}\\
\text{[}Anzahl der Punkte in einer Zeile, d.h. in x-Richtung, Zahlenformat: \emph{int}]\\
\text{[}Anzahl der Punkte in einer Spalte, d.h. in y-Richtung, \emph{int}]\\
\text{[}Anzahl der aufzulegenden Ebenen, \emph{int}]\\
\text{[}Relative Position der Ebenen zueinander, in z-Richtung, \emph{double, negativ}]\\
\text{[}$x_{w1}\ y_{w1}$, \emph{double, double, durch Leerzeichen getrennt}]\\
:\\
\text{[}$x_{wn}\ y_{wn}$]\\

Der vierte Parameter, die relative Position der Ebenen zueinander, entspricht der Dicke einer Glasplatte. Die Anzahl der Punkte $n$ ist gleich dem Wert, den man durch Multiplikation der ersten drei Parameter erh�lt. Die x- und y-Koordinaten der einzelnen Punkte bezeichnen ihre Lage in der xy-Ebene des Weltkoordinatensystems. Die Einheit ist [mm]. Die Punkte sind zeilenweise sortiert einzugeben, beginnend mit dem ersten Punkt der obersten Zeile.\\

\emph{Beispiel:}
\begin{quote}
11\\
9\\
5\\
-10.0\\
0.0 0.0\\
5.0 0.0\\
10.0 0.0\\
15.0 0.0\\
:\\
0.0 5.0\\
5.0 5.0\\
10.0 5.0\\
15.0 5.0\\
:\\
45.0 40.0\\
50.0 40.0\\\\
\end{quote}

\textbf{Projektorkalibrierung}, Datei \verb$world_projector.txt$\vspace{11pt}\\
\text{[}Anzahl der aufzulegenden Ebenen, \emph{int}]\\
\text{[}Relative Position der Ebenen zueinander, in z-Richtung, \emph{double, negativ}]\\


Der zweite Parameter, die relative Position der Ebenen zueinander, entspricht der Dicke einer Glasplatte.

\emph{Beispiel:}
\begin{quote}
5\\
-10.0\\
\end{quote}

