 \cleardoublepage
\chapter{Conclusion}

\section{Summary}

With the use of SDN and OpenFlow, a framework for creating programmable networks already exists. However the integration of an SDN Controller can further extend the capabilities of SDN in OpenStack. By default, the network bandwidth supplied to different servers is on a best-effort basis and topologies that are deployed using the orchestration service Heat cannot be placed on specific hosts. The work described in this thesis explains how these limitations can be overcome. With the help of the Open vSwitch configuration tool it is possible to get information about the current state of the switched network and enable QoS using Queues. The required configuration and information can now be accessed through an API. The Elastic Media Manager is able to access this information and make decisions about where the topology has the best-performing network connectivity. The bandwidth rates according to their Service-Level Agreements can be easily updated and extended. The evaluation gives a step-by-step explanation of the improvements in bandwidth rates with the use of the Connectivity Manager in relation to the requirements.

\section{Problems Encountered}

One significant problem during the development process was getting a stable test-bed running with the correct configuration. Devstack helped significantly in this regard, because aside from adjusting the configuration file, not many manual corrections needed to be made. However some of the features that are in the 'stable' version of OpenStack Juno still need further bug-fixing and cannot currently be used in a production environment. During the testing of various state-of-the-art solutions, a significant amount of time was spent on testing different versions, which was unsuccessful. Due to the significant time taken, comparisons to other solutions for enabling Quality of Service could not be evaluated. 

\section{Future Work}

For future development, the algorithm for selecting the best-performing host could take more dynamic factors into account. One possibility is to check the network speed of all the host's network interfaces. Furthermore, it would be a good approach to implement the features of the Connectivity Manager Agent as a Neutron extension and make use of the existing Neutron API. The implementation of QoS could be based on the existing blueprint and be shared with the OpenStack community to achieve an integration into the Neutron repository.

Additional, one further step that could be evaluated for Quality of Service is to perform Flow modifications, as opposed to the currently used procedure of attaching the Queue to the OVS Port and having a rate-limit on the egress traffic. With this approach, the ingress traffic for a specific Port could be filtered with enqueuing and then set as the Flow action.  When multiple Flow entries are in use, a more granular selection than just Ports can be applied by filtering for match fields such as IP and MAC addresses.