\chapter{List of source codes}
\index{a2ps}\index{Pretty Printer}\index{Quelltexte}

\textcolor{darkred}{Anmerkungen: Quelltextausz�ge zu einer Implementierung sind im Anhang dann sinnvoll, wenn einige, spezielle Implementierungstechniken aufgezeigt werden sollen, die in der Darstellung als Algorithmus oder Pseudocode nicht deutlich werden. Keinesfalls soll der gesamte Quelltext angeh�ngt werden und weiterhin soll auch in einer Vorbemerkung die Auswahl der Quelltextausz�ge genau erkl�rt werden.}

\textcolor{darkred}{Verwendet wird das freie Quelltext-Pretty-Printing-Tool a2ps.exe mit folgender Aufrufkonvention (vgl. \mbox{[a2ps 07, grep 07]):}}


\verb$     a2ps.exe --pretty-print=cxx -i test.cpp -o test.ps -T3$

\medskip
\medskip

\textcolor{darkred}{Die Quelltextdateien d�rfen hierf�r eine Zeilenl�nge von 80 nicht �berschreiten. Die st�renden Kommentare im Header und Footer des entstandenen .ps-Files k�nnen mithilfe des freien Tools grep.exe automatisiert entfernt werden. Eine Batch-Datei f�r den gesamten Konvertierungsprozess inklusive Konvertierung in das pdf-Format hat beispielsweise folgenden Inhalt:}


\verb$     a2ps --pretty-print=cxx -i %1 -o tmp -T3$\\
\verb$     grep -v "Gedruckt von" tmp | grep -v ") footer" > %1.ps$\\
\verb$     del tmp$\\
\verb$     "c:\programme\adobe\acrobat 7.0\Acrobat\acrobat.exe" %1.ps$\\


