\chapter{Evaluation}

\section{Feature analysis}



\section{Connectivity Manager integration with NUBOMEDIA project}

% % CM deliverable % %

The Connectivity Manager (CM) is part of the NUBOMEDIA platform and is placed between the virtual network resource management of the cloud infrastructure and the
multimedia application. The main focus of the CM is related to management and control of network functions of the virtual network infrastructure provided by OpenStack.

Nubomedia is an elastic Platform as a Service (PaaS) cloud for interactive social multimedia. Its architecture is based on media pipelines: chains of elements providing media capabilities such as encryption, transcoding, augmented reality or video content analysis. These chains allow building arbitrarily complex media processing for applications. As a unique feature, from the point of view of the pipelines, the NUBOMEDIA cloud infrastructure behaves as a single virtual super-computer encompassing all the available resources of the underlying physical network.


% % from EMM % %


Deployment
The deployment phase starts when an administrator requests the instantiation of
NUBOMEDIA. Inside the appropriate request, the administrator specifies what are the
capabilities required for this instance of the platform. Based on the information
received, the EMM needs to instantiate virtual resources on the Virtualized
Infrastructure. For this step there are several ways of doing it:
-Request directly OpenStack virtual networks and compute resources
-Request to Heat the required resources creating a template
For simplifying and making homogenous the whole process, we selected the second
option. The EMM has to create a Heat template based on the required resources which
are requested by the admin. Once the template is created, it is sent to Heat using its API.
At the end of the deployment process, Heat sends back all the information related with
the instantiated resources which the EMM can use for moving to the next phases.


Runtime
Once all the NUBOMEDIA components are deployed and started, it is needed to
actively check which specific levels of SLAs are met. The EMM provides a runtime
system which actively controls the situation of a group of NUBOMEDIA components,
and based on some policies, which the administrator inserted, decides whether to scale
in or out them.
When the administrator requests the instantiation of a NUBOMEDIA platform, it puts
also specific policies to specific elements. A policy is just a set of alarms and actions.
There are two different ways of realizing such system via triggering mechanisms:
- Actively: the EMM actively checks whether the conditions of the alarm are met.
For doing it, it typically requests to the monitoring system last values of the
metrics involved, and performs some basic operations for evaluating the status.
- Passively: the EMM pushes the alarm into the monitoring system. When the
conditions are met, the monitoring system triggers the alarm to the EMM.
For NUBOMEDIA Release 3, it was decided to use the first approach. Once the EMM
realizes that an alarm is in ACTIVE state, it has to trigger the execution of the action
contained in the Policy. Typically this policy can involve the instantiation/removal of
the NUBOMEDIA elements.
\section{Conclusion}