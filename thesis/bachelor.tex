%%%%%%%%%%%%%%%%%%%%%%%%%%%%%%%%%%%%%%%%%%%%%%%%%%%%%%%%%%%%%%%%%%%%%%%%%%%%%%
%
% Vorlage f�r eine Diplomarbeit f�r den Fachbereich 
% Mikrosystemtechnik/Mikroelektronik HTW Berlin
% Prof. Dr. Peter Gregorius
%
%%%%%%%%%%%%%%%%%%%%%%%%%%%%%%%%%%%%%%%%%%%%%%%%%%%%%%%%%%%%%%%%%%%%%%%%%%%%%%



\documentclass
[a4paper,english,oneside,openright,                    % Kap.beginn immer rechts! (fkt. nur bei report, nicht bei article)
%twoside statt oneside wenn beidseitig gedruckt wird
11pt                          % ersatzweise 12pt, wenn mehr Seiten entstehen sollen
]
{report}



\usepackage[latin1]{inputenc} % Zeichensatz, erm�glicht die direkte Eingabe von Umlauten im Editor
\usepackage[pdftex]{graphicx} % Einbindung von Grafiken (pdf, png, jpg)
\usepackage{float}            % bietet Option [H] f�r bombenfestes Verankern
\usepackage[english]{babel}   % Silbentrennung nach der neuen deutschen Rechtschreibung, z.B.: Sys-tem
\usepackage{amstext}          % f�r Klartext via \text{} in Formeln
\usepackage{amsfonts}         % f�r komplexere Formeln (Mengensymbole ...)
\usepackage{amssymb}          % f�r komplexere Formeln (Mengensymbole ...)
\usepackage{bm}               % bold math, f�r \bm{}
\usepackage{enumerate}        % verbessert Aufz�hlungen
\usepackage[bottom]{footmisc} % Fussnoten am Seitenende
\usepackage{array}            % f�r Tabellen: bindet tabular-Umgebung ein
\usepackage{tabularx}
\usepackage{algorithm}        % f�r Algorithmen
\usepackage{algorithmic}      % f�r Algorithmen
\usepackage{listings}
\usepackage{ntheorem}
\usepackage{theorem}
\usepackage{caption}
\usepackage[acronym]{glossaries} % fuer Glossar mit Acronymen]
\usepackage{pdfpages}         % f�r die Einbindung kompletter pdf-*Seiten*
\usepackage{parskip}          % zw. Abs�tzen: eine knappe Leerzeile statt h�ngender Einz�ge
%\usepackage[right]{eurosym}   % Eurosymbol
\usepackage{xcolor}           % farbiger Text
\usepackage[hyphens]{url}     % f�r \url{http://www}, Option hyp erlaubt auch Umbruch nach "-"
\usepackage{makeidx}          % Package zur Indexerstellung
\usepackage{multicol}         % zur Indexerstellung in zwei Spalten
\usepackage[numbers, square]{natbib}   % F�r \setlength{\bibsep}{3mm}; square macht eckige Klammern


%\usepackage[cmex10]{amsmath}  % f�r erw. Formeloptionen, Option [] zur Vermeidung von Type3-Fonts
%\usepackage{mathcomp}         %\tcmu \tcohm \tccelsius.. im Mathemodus, nichtkursiv; problematisch!
%\usepackage{textcomp}         % f�r \textdegree , \textcelsius , macht aber manchmal auch Probleme!


%\usepackage[plainpages=false, hypertexnames=false]{hyperref}
                               % hyperref statt \url geht, vertr�gt sich allerdings nicht mit den 
                               % eingef�gten / ver�nderten Seitenzahlen ... (die stimmen dann nicht mehr)

\usepackage{hyperref}
\hypersetup{
    colorlinks,
    citecolor=black,
    filecolor=black,
    linkcolor=black,
    urlcolor=black
}
                                
\definecolor{darkred}{rgb}{0.7,0.0,0.0}

\sloppy                       % gro�z�giger Zeilenumbruch 
                             % -> keine rechts rausragenden Zeilen mehr


\renewcommand{\listofalgorithms}   % Text "Algoverzeichnis", statt "List of Algos"
{
\begingroup
\listof{algorithm}{List of Algorithms}
\endgroup
}

% Include numbering for subsubsections
\setcounter{secnumdepth}{4}
%\setcounter{tocdepth}{3}



%%%%%%%%%%%%%%%%%%%%%%%%%%%%%%%%%%%%%%%%%%%%%%%%%%%%%%%%%%%%%%%%%%%%%%%%%%%%%%
%
% Index-Erstellung
%
% Anmerkung: f�r die Indexerstellung muss auch die TeXnicCenter-IDE angepasst 
% werden:
% 1.) Projekt / Eigenschaften / verwendet MakeIndex [x] 
% 2.) Ausgabe / Ausgabeprofil definieren
%
\makeindex % erstelle einen Index bzw. ein Sachverzeichnis)
%
% Wenn kein Index gew�nscht ist: einfach \makeindex auskommentieren
%
%%%%%%%%%%%%%%%%%%%%%%%%%%%%%%%%%%%%%%%%%%%%%%%%%%%%%%%%%%%%%%%%%%%%%%%%%%%%%%



%%%%%%%%%%%%%%%%%%%%%%%%%%%%%%%%%%%%%%%%%%%%%%%%%%%%%%%%%%%%%%%%%%%%%%%%%%%%%%
%
% Literaturverzeichnis mit BibTeX
%
\bibliographystyle{ka-style} % Uni-KA-Style  
\setlength{\bibsep}{3mm}                  % Abst�nde im Litverzeichnis
%
% Anmerkung: das Dokument enth�lt _zwei_ Literaturverzeichnisse
% 1.) Ein auf Basis von bibliografie.bib mit BibTeX erstelltes Verzeichnis
% 2.) Ein einfach getipptes Verzeichnis: Literatur.tex
%
% -> das nicht gew�nschte einfach auskommentieren
%
%%%%%%%%%%%%%%%%%%%%%%%%%%%%%%%%%%%%%%%%%%%%%%%%%%%%%%%%%%%%%%%%%%%%%%%%%%%%%%



%%%%%%%%%%%%%%%%%%%%%%%%%%%%%%%%%%%%%%%%%%%%%%%%%%%%%%%%%%%%%%%%%%%%%%%%%%%%%%
%
% Gr��enanpassungen
%
\setlength{\unitlength}{1cm}
\setlength{\oddsidemargin}{0.3cm}
\setlength{\evensidemargin}{0.3cm}
\setlength{\textwidth}{15.5cm}
\setlength{\topmargin}{-1.2cm}
\setlength{\textheight}{23cm}
\columnsep 0.5cm
%
%%%%%%%%%%%%%%%%%%%%%%%%%%%%%%%%%%%%%%%%%%%%%%%%%%%%%%%%%%%%%%%%%%%%%%%%%%%%%%



%%%%%%%%%%%%%%%%%%%%%%%%%%%%%%%%%%%%%%%%%%%%%%%%%%%%%%%%%%%%%%%%%%%%%%%%%%%%%%
%
% Beispiel f�r die Anpassung des Satzspiegels und 
% die Verwendung von Schnittmarken (momentan ausgeschaltet)
% Im Beispiel: Anpassung des Drucks auf Taschenbuchformat 
%
% Obacht: f�r Tests hiermit (Probeausdrucke...): 
% stets im Adobe Acrobat im Druckdialog die Seitenanpassung *abschalten*!
% sonst stimmen die Ma�e nicht!
% 
%
%\usepackage[total={90mm,144mm},centering]{geometry}
%\geometry{papersize={120mm,190mm}} 
%\usepackage[a4,cam,center]{crop}
%\crop[]
%
% Schnittmarken und Satzspiegel - Ende
%
%%%%%%%%%%%%%%%%%%%%%%%%%%%%%%%%%%%%%%%%%%%%%%%%%%%%%%%%%%%%%%%%%%%%%%%%%%%%%%



%%%%%%%%%%%%%%%%%%%%%%%%%%%%%%%%%%%%%%%%%%%%%%%%%%%%%%%%%%%%%%%%%%%%%%%%%%%%%%
%
% Abk�rzungsliste, Liste explizit vorgegebener Abk.
%
% Anmerkung: f�r W�rter mit Umlauten
% muss das Paket \usepackage[T1]{fontenc} eingebunden werden --
% in der vorliegenden Version funktionieren *keine* Umlaute!!
% 
\hyphenation{Samm-lung-en Samm-lung Stau-beck-en Vor-na-me-in-i-ti-al % Ver-st\"ar-ker-aus-gang 
Nach-na-me Kurz-be-zeich-nung deutsch-spra-chige deutsch-sprachig Screen-shot Screen-shots schluss-end-lich Schluss-end-lich Make-In-dex Da-tei-name Da-tei-namen Ur-instinkt Ur-instinkte} 
%
%%%%%%%%%%%%%%%%%%%%%%%%%%%%%%%%%%%%%%%%%%%%%%%%%%%%%%%%%%%%%%%%%%%%%%%%%%%%%%



\begin{document}
\pagestyle{empty}
\begin{titlepage}
\begin{figure}
  \begin{center}
    \hbox to \hsize{%
      \begin{tabular}[m]{c}
        \includegraphics[width=6.5cm]{BilderAllgemein/unilogo.png}
      \end{tabular}
      \hfill%
      \begin{tabular}[m]{c}
        Studiengang Computer Engineering \\
				Fachbereich 1\\
        Prof. Dr. Thomas Baar\\
      \end{tabular}%
    }
  \end{center}
\end{figure}

\begin{center}
\rule{0pt}{0pt}
\vfill
\vfill
\vfill
\vfill

\begin{huge}
Design and implementation of a connectivity manager for virtual scalable network environments\\[0.75ex]
\end{huge}

\vfill
\vfill

Bachelorarbeit\\ von\\

\vspace*{.5cm}
Manuel Bergler\\
\vspace{.5cm}
01.12.2014 -- 08.02.2015 \\

\vfill
\vfill
\vfill
\vfill

\begin{tabular}{rl}
Referent:   & Prof. Dr. Thomas Baar\\
Korreferent:   & Prof. Dr. Thomas Baar\\

Betreuer:   & Benjamin Reichel M. Sc.\\
\end{tabular}
\end{center}
\end{titlepage}



\newpage



\text{ }
\vspace{13.5cm}


Manuel Bergler\\
Urbanstr. 26\\
10967 Berlin\\

Hiermit versichere ich, dass ich die von mir vorgelegte Arbeit selbstst�ndig verfasst habe, dass ich die verwendeten Quellen, Internet-Quellen und Hilfsmittel vollst�ndig angegeben habe und dass ich die Stellen der Arbeit -- einschlie�lich Tabellen, Karten und Abbildungen~--, die anderen Werken oder dem Internet im Wortlaut oder dem Sinn nach entnommen sind, auf jeden Fall unter Angabe der Quelle als Entlehnung kenntlich gemacht habe.\\

Berlin, den 08. Februar 2015\\
\medskip
\medskip

(Unterschrift)\\
\underline{~~~~~~~~~~~~~~~~~~~~~~~~~~~~~~~~~~~~~~~~}\\
Manuel Bergler\\



\newpage

               % und eidesstattliche Erkl�rung


%%%%%%%%%%%%%%%%%%%%%%%%%%%%%%%%%%%%%%%%%%%%%%%%%%%%%%%%%%%%%%%%%%%%%%%%%%%%%%



\pagestyle{plain}
\pagenumbering{arabic}
\setcounter{page}{3}
\tableofcontents
\cleardoublepage



%%%%%%%%%%%%%%%%%%%%%%%%%%%%%%%%%%%%%%%%%%%%%%%%%%%%%%%%%%%%%%%%%%%%%%%%%%%%%%
%
% Anmerkung:
%
% Falls Style "article" gew�nscht ist und
% und dennoch die Kapitel auf der rechten Seite beginnen sollen,
% dann ist nach jedem \include einzuf�gen:
% \cleardoublepage
%
%%%%%%%%%%%%%%%%%%%%%%%%%%%%%%%%%%%%%%%%%%%%%%%%%%%%%%%%%%%%%%%%%%%%%%%%%%%%%%




\listoffigures
\protect \addcontentsline{toc}{chapter}{List of figures}
\cleardoublepage


\listoftables
\protect \addcontentsline{toc}{chapter}{List of tables}
\cleardoublepage


\listofalgorithms
\protect \addcontentsline{toc}{chapter}{List of algorithms}
\cleardoublepage



\chapter{Introduction}
\label{chapter_introduction}



\section{Motivation}

The demands on networks have changed dramatically in the past two decades, with an ever-growing number of people and devices relying on interconnected applications and services. The underlying infrastructure has been left mostly unchanged and is approaching its limits. In order to resolve this, Software Defined Networking (SDN) is going to be extending and replacing parts of traditional networking infrastructures. SDN separates the network into control and forwarding planes and therefore allows a more efficient orchestration and automation of network services.

The use of cloud-based services, with not only competitive pricing but also high-availability and fast network access,  is taking over traditional self-hosted data centers. The ease of administration and deployment of new Virtual Machines (VMs) on the fly make it possible to effortlessly create a topology of servers running different services.

Network services have different requirements, depending on the type of data and their importance. The classification of network traffic can be done through Quality of Service (QoS). A new approach has to be made to enable the use of QoS through an API in virtualized cloud infrastructures like OpenStack, to achieve controlled traffic right from the deployment of Virtual Machines on.


\section{Network Architecture}
Today's traffic patterns, the rise of cloud computing and "big data" to only name a few examples, are exceeding the capacity of classic network architectures. With scalable computing and storage the common-place tree-structured network infrastructure with Ethernet switches are not efficient and manageable enough. 

The increasing complexity of problems that have to be faced in networks and the need to control network traffic through software, are only a selection of the reasons why the Open Networking Foundation (ONF) developed an approach called Software-Defined Networking (SDN).

SDN is a leading-edge approach where the network control is separated from the forwarding functions. The centralized network intelligence allows programming the network, without a need to access the underlying infrastructure. Therefore a shift of today's networks to more flexibility, programmability and scalability is going to take place.

\section{Objective}

The primary objective of this work is the development of a network orchestrator which is able to apply Quality of Service to the network interfaces of Virtual Machines. These Virtual Machines are deployed with OpenStack Nova and connected to an Open vSwitch, which uses OpenFlow. Another task of the Connectivity Manager is to select which OpenStack hypervisor new VMs should be running on, which takes different runtime parameters into account. The CM should be able to be applied in environments with scalable hypervisors and VMs.

\section{Scope}
The scope of this work includes a Connectivity Manager which will have a Connectivity Manager Agent running on the cloud controller within the OpenStack infrastructure, to provide access to the hypervisors of OpenStack Nova. These two components have to be implemented and integrated with the existing Open vSwitches. As a reference for a cloud infrastructure, multimedia communications like the Nubomedia project will be used. The deployment of this cloud is then tested on different performance characteristics like network bandwidth, latency, CPU utilization and memory usage.

In virtualized cloud infrastructure like OpenStack, the placement of Virtual Machines (VMs) on a particular compute node can be decided on by comparing different run-time parameters. The network connectivity between those VMs has to be prioritized and classified into different classes, depending on the service that are running on it.

Currently there are a number of solutions for managing network connectivity between VMs. A comparison and their current limitations follows in the next section. The chosen approach is to extend the existing network control and management services with Quality of Service (QoS) capabilities. In support of the thesis the Connectivity Manager will be implemented and the differences in bandwidth usage will be shown in one use-case.

\section{Overview}

\textbf{Chapter 1} begins with the motivation for this thesis and gives a brief introduction into the objectives and the scope.

\textbf{Chapter 2} gives an overview of traditional network concepts and a introduction to SDN and its components. Furthermore the different services that make up OpenStack will be described.

\textbf{Chapter 3} conceptualizes the state-of-the-art solutions that are currently available and evaluates their implementation and limitations.

\textbf{Chapter 4} contains an analysis of requirements and an architectural overview of the Connectivity Manager. Moreover design aspects are introduced and illustrated according to their requirements.

\textbf{Chapter 5} examines the implementation of the Connectivity Manager and Agent.

\textbf{Chapter 6} evaluates the network performance tests on the basis of a particular use-case.

\textbf{Chapter 7} summarizes the results of this work and gives an overview on possible future work.



\chapter{Fundamentals and Related Work}

\section{Networking}

\subsection{Traditional network architectures}

\subsubsection{IP based communication}
TCP/IP
IPv4
TCP / UDP traffic

\subsubsection{Switches and routers}

\subsubsection{Quality of Service}

\subsection{SDN architectures}

\subsubsection{Software-defined network concept}

\subsubsection{OpenVSwitch}

\subsubsection{OpenFlow}

\section{Cloud computing infrastructure}

\subsubsection{OpenStack}

\subsubsection{Devstack}

\subsubsection{OpenStack Nova}

\subsubsection{OpenStack Neutron}

\section{Conclusion}
\chapter{Requirements}

\section{Functional Requirements}

This section identifies the functional requirements of the Connectivity Manager, specifically as needed for the NUBOMEDIA use-case.

\subsection{Service-Level Agreement Enforcement}

One of the key objectives of the Connectivity Manager is to grant different Service-Level Agreements (SLA) to the links between Virtual Machines. The agreement is set as Quality of Service assurances with the minimal and maximum bandwidth rate set. Network performance problems can provide a negative experience for the end-user, as well as productivity and economic loss. This is why some services need to have an ensured premium traffic.

\subsection{Optimal Virtual Machine Placement}

The placement of Virtual Machines makes a tremendous difference in terms of their network and overall resource performance. It needs to be evaluated which placement makes for the best-available network bandwidth between VMs within the internal network. The current utilization of hosts need to be taken into account as well. A stack should only be deployed if the resources are available at the time of deployment without overcommitting any hardware resources.

\subsection{Integration with Elastic Media Manager}

The Connectivity Manager needs to integrated with the Elastic Media Manager (EMM) which is used for deploying a topology of resources within a cloud infrastructure. Furthermore it provisions the instances and manages them during their runtime through services such as upscaling the amount of instances after certain utilization alarms are triggered. The CM communicates with the EMM in order to enable the two previously-mentioned requirements for the overall platform.

\section{Non-functional Requirements}

Non-functional requirements generally specify criteria to do with the operation of a system and not with its behavior. Thus the Connectivity Manager should also fit the following characteristics.

\subsection{Scalability}

Today's data-centers can grow in a fast-pace, especially in connection with automated up-scaling of compute resources at a certain level of utilization. This is why the underlying virtualized network software needs to be scalable too.

\subsection{Modularity}

Building modular software not only simplifies further development for a third-party, but also makes it easier to exchange certain parts of the software for improvements or maintenance at a later date. The separation into two different components with a defined API makes it more flexible.

\subsection{Interoperability}

In the case of the use of Open vSwitch, interoperability is given because of the availability for various architectures. The integration into the Linux Kernel and the use of standardized protocols such as OpenFlow are a significant factor.
\chapter{State of the art}

\section{Overview}

Three existing solutions for extending Neutron with additional SDN features have been tested. They were selected based on the requirements given in section 3.

\section{OpenDaylight SDN controller}

% %https://www.openstack.org/assets/presentation-media/osodlatl.pdf % %
% %https://github.com/opendaylight/docs/blob/master/manuals/developers-guide/src/main/asciidoc/ovsdb.adoc % %

OpenDaylight is fully implemented in Java. The Controller platform has multiple Northbound \& Southbound interfaces. OpenDaylight exposes a single common OpenStack Service Northbound API which exactly matches the Neutron API. The OpenDaylight OpenStack Neutron Plugin simply passes through and therefore pushes complexity to OpenDaylight and simplifies the OpenStack plugin. The ML2 mechanism driver in Neutron has to be set to the OpenDaylight ML2 plugin with the ODL agent running on the Compute Nodes. The OpenDaylight controller can be run on the Control Node or on a separate VM. The Open vSwitch database (OVSDB) Plugin component for OpenDaylight implements the OVSDB management protocol that allows the southbound configuration of vSwitches. The OpenDaylight controller uses the native OVSDB implementation to manipulate the Open vSwitch database. The component comprises a library and various plugin usages. The OVSDB protocol uses JSON/RPC calls to manipulate a physical or virtual switch that has OVSDB attached to it.

\begin{figure}[H]
\centering
\includegraphics[width=0.9\textwidth]{images/sota/odl_architecture.png}
\caption{Architecture of OpenDaylight Virtualization edition}
\end{figure}

The OVSDB component is accessible through a Northbound ReST API, which enables the operator to connect to the OpenFlow controller and modify various OVSDB tables. Through this API QoS rules can be deployed. Because it connects directly to the OpenVSwitch tables, all the QoS types that come with OpenVSwitch can be deployed (DSCP marking, setting priority, min-/max-rate for switch ports \& OpenFlow Queues). In the local testbed we were able to successfully deploy QoS rules on the ports of Virtual Machines. 


\section{Ryu SDN controller}

% %https://nsrc.org/workshops/2014/nznog-sdn/raw-attachment/wiki/WikiStart/Ryu.pdf % %
% %https://wiki.openstack.org/wiki/Neutron/OFAgent/ComparisonWithOVS % %

Ryu is a component-based software defined networking framework which fully supports OpenFlow 1.0, 1.2, 1.3 and 1.4 switches and is fully written in Python. Ryu is a full featured OpenFlow controller that supports GRE and VLAN tunnelling. The OpenFlow controller that is embedded in the agent sets Flows on the switch by sending OpenFlow messages to the switch. It includes a set of apps which build the base of the SDN controller like L2 switch, ReST interface, topology viewer and tunnel modules. Ryu also includes an app that allows to set QoS rules through a ReST interface which uses a OVSDB interaction library to apply those. The QoS rules can be either applied to a specific Queue within a VLAN or a Switch port. It supports DSCP tagging and setting the min-rate and max-rate of an interface.

\begin{figure}[H]
\centering
\includegraphics[width=0.7\textwidth]{images/sota/ryu_architecture.png}
\caption{Ryu architecture}
\end{figure}


As of OpenStack IceHouse Ryu has been renamed to OFagent and is included in the Neutron repository. In order to use it as the SDN framework for OpenStack Neutron, OFagent has to be set as both the ML2 mechanism driver (running on the Control / Network node) and the Neutron agent (running on the Compute node). 


\section{OpenStack Neutron - QoS Extension}

% %https://wiki.openstack.org/wiki/Neutron/QoS % %
A Neutron extension has been partially implemented for OpenStack IceHouse which includes an API for setting and retrieving QoS on a per-tenant and per-port basis.

% -> further describe the structure of the implementation !! %

\begin{figure}[H]
\centering
\includegraphics[width=0.3\textwidth]{images/sota/neutron_qos_extension.png}
\caption{Neutron QoS Extension architecture}
\end{figure}



\section{Problem statement}

%Show why previously mentioned projects fail for our requirements, analyse and compare them. Make table with features. Also state that they wouldn't include the VM placement feature. %
This section lists the restrictions that have been discovered with the previously mentioned solutions which further strengthens the motivation for the implementation of the Connectivity Manager. A feature comparison and analysis is given.

\textbf{Problems encountered ODL}
The local testbed used for the integration of OpenStack Juno and OpenDaylight Helium consisted of 2 hosts, one running the OpenStack control node and OpenDaylight Controller and a OpenStack Compute Node on the second host. During the tests it was not possible to get the public network access for the Virtual Machines working. This and the fact that it ODL is very complex to debug and understand all underlying processes led us to the decision not to use OpenDaylight.

\textbf{Problems encountered Ryu}
The test of Ryu was unsuccessful due to a number of errors while stacking the test environment using Devstack. It was not possible to launch instances and test the QoS features. The lack of proper documentation for the interaction with OpenStack Neutron led us to look more into other SDN controllers for our particular use case. Currently
Ryu doesn't support the Distributed Virtual Routing feature that has been introduced with OpenStack Juno.

\textbf{Problems encountered Neutron QoS Extension}
The implementation has not been finished and merged into Neutron, however the basic deployment of QoS seem to have been tested successfully. 

At the moment it is not clear if the OpenStack community will keep working on this, according to the whiteboard it was deferred to Juno, but it's not included in the current release and no active development is stated in the code / documentation platforms of the OpenStack community.

The patch consists of an extension to the Neutron API which allows setting QoS rules through the Neutron Python client, the actual Neutron extension with the QoS, QoS Driver in the OpenVSwitch agent and an addition to the Neutron Database that includes QoS.


\chapter{Design}

\section{Architecture overview}

The Connectivity Manager is logically located between the EMM and the cloud infrastructure and provides the following two functionalities:
\begin{itemize}
\item \textbf{Optimal Instance Placement:} During the deployment of a stack an algorithm chooses where individual instances are placed within the cloud infrastructure.
\item \textbf{Service-Level-Agreement enforcement:} Depending on the services that an instance provides to the rest of the stack, certain requirements for its network performance need to be fulfilled.
\end{itemize}

\begin{figure}[H]
\centering

\includegraphics[width=0.5\textwidth]{images/design/functional_architecture}

\caption{High-level architecture of the Connectivity Manager}
\end{figure}

The \textit{Instance Placement Engine} determines if and where the instances should be deployed. It does so by comparing the current utilization and capacity of the available compute nodes within the availability zone.

The \textit{QoS Manager} enforces different QoS policies based on the type of service that the instance is grouped in. A guaranteed and maximum bit-rate for the network port of an instance can be set. This way a certain network performance can be insured.

\section{Connection between Manager \& Agent}

The Connectivity Manager and Agent are two separate applications that communicate using a ReST API. 

\begin{figure}[H]
\centering

\includegraphics[width=0.6\textwidth]{images/design/modular_architecture_cm_cma}

\caption{Minimized architecture of the Connectivity Manager and its integrations}
\end{figure}

This design was chosen first of all because the Connectivity Manager is integrated in the EMM, which is required to be placed anywhere outside of the data center. Second of all the Connectivity Manager Agent needs to  the OVSDB on the compute nodes and consequently needs to be within the internal management network of the OpenStack infrastructure.

The sequence diagram below displays the work-flow that the CM passes during the run-time.

\begin{figure}[H]
\centering

\includegraphics[width=0.9\textwidth]{images/design/sequence_diagram}

\caption{Workflow: Deployment of stack \& Assignment of QoS policies}
\end{figure}

As can be seen, the Connectivity Manager receives the Topology that contains a description of the configuration and specifications of the whole cloud. For the placement decision the CM to needs to get the information about the current state of the infrastructure. This exchange with the CM Agent occurs through the given API. Upon reception of that data, the placement algorithm sets the availability zone for each instance within the topology. The topology is then converted into a Heat template by the Template Manager. Once the template got deployed by the Heat Client a runtime agent starts. The purpose of the runtime agent is to continuously check the state of the stack. Once the stack has reached the 'DEPLOYED' state, the runtime agent requests the CM to set the QoS policies according to previously configured values. This configuration is subsequently transmitted to the CM Agent whose task is then to enable it on the according ports of the instances within the Open vSwitch.

\section{Design of Connectivity Manager}


\begin{figure}[H]
\centering

\includegraphics[width=0.9\textwidth]{images/design/cm_topology_object}

\caption{Deployment: Topology object from EMM}
\end{figure}


\subsection{Algorithm for Instance Placement}



\newpage
\section{Design of Connectivity Manager Agent}

\begin{figure}[H]
\centering

\includegraphics[width=0.7\textwidth]{images/design/cm_agent_class_diagram}

\caption{Class diagram: Connectivity Manager Agent - Core package}
\end{figure}


\begin{figure}[H]
\centering

\includegraphics[width=0.7\textwidth]{images/design/activity_host_list}

\caption{Activity diagram: Get list of hosts}
\end{figure}

\begin{figure}[H]
\centering

\includegraphics[width=0.7\textwidth]{images/design/activity_set_qos}

\caption{Activity diagram: Set QoS rates for all servers}
\end{figure}

\begin{figure}[H]
\centering

\includegraphics[width=0.7\textwidth]{images/design/activity_get_ovs_port_server}

\caption{Activity diagram: Get OVS Port ID for server}
\end{figure}

\begin{figure}[H]
\centering

\includegraphics[width=0.7\textwidth]{images/design/activity_get_qos_id_for_ovs_port}

\caption{Activity diagram: Get QoS ID for OVS Port}
\end{figure}





\section{Conclusion}
\chapter{Implementation}


\section{Environment}

The software developed in this thesis is completely realized using the Python programming language. This choice was made because OpenStack offers Python clients that connect to their API and the Elastic Media Manager is also programmed in Python.

PyCharm was selected as the integrated development environment (IDE) to simplify the programming and testing lifecycles. The code is under revision control using git and the repository that contains both the code for the CM and CM Agent consists of two main branches: master and develop. The develop branch holds the latest changes and upon successful testing those were merged back into master, which is always in a production-ready state.

\subsubsection{Project Structure}

The code is separated into two different projects in order to allow testing of the integration concurrently. The following graphic outlines the focus within the Elastic Media Manager (EMM), whilst the CM Agent is separated and has its own structure.

\begin{figure}[H]
\centering

\includegraphics[width=0.9\textwidth]{images/implementation/cm_implementation_focus_overview}

\caption{Implementation focus}
\end{figure}

The structure for the CM is dictated by the already existing implementation of the EMM and the scope of this thesis includes solely its extension with a Connectivity Manager interface and service plus the needed changes in the other interfaces to make use of the methods within the CM.

The Connectivity Manager contains the ReST API, the Agent core, clients and other helper classes.

\subsubsection{Local OpenStack Test Environment}

In order to test the Connectivity Manager Agent and the use of its OpenStack API clients a testbed was installed. This test-bed was set up using Vagrant, as it allows to start virtual machines from the command-line and can easily be provisioned and managed. In order to test the software across multiple compute nodes a set up with 2 VM's was installed.

For the installation of OpenStack the devstack script was used, which takes care of not only the deployment of the different components but also their configuration. The configuration parameters are set in the 'local.conf' file. For the OpenStack cluster controller the following configuration was used:

\begin{lstlisting}[language=json]
[[local|localrc]]
ADMIN_PASSWORD=pass
DATABASE_PASSWORD=pass
RABBIT_PASSWORD=pass
SERVICE_PASSWORD=pass
SERVICE_TOKEN=a682f596-76f3-11e3-b3b2-e716f9080d50

HOST_IP=192.168.120.15
OVS_PHYSICAL_BRIDGE=br-ex
MULTI_HOST=1

# Enable Logging
LOGFILE=/opt/stack/logs/stack.sh.log
VERBOSE=True
OFFLINE=True
RECLONE=no
LOG_COLOR=True
SCREEN_LOGDIR=/opt/stack/logs

# Neutron
disable_service n-net
enable_service q-svc
enable_service q-agt
enable_service q-dhcp
enable_service q-l3
enable_service q-meta

# OpenStack API paths
MYSQL_HOST=192.168.120.15
RABBIT_HOST=192.168.120.15
GLANCE_HOSTPORT=192.168.120.15:9292
KEYSTONE_AUTH_HOST=192.168.120.15
KEYSTONE_SERVICE_HOST=192.168.120.15

IMAGE_URLS="$IMAGE_URLS,http://cloud-images.ubuntu.com/releases/trusty/
release/ubuntu-14.04-server-cloudimg-amd64-disk1.img"
\end{lstlisting}

The configuration file for the second node, which solely runs Nova, the Open vSwitch agent and the Rabbit MQ is identical except for its enabled services and host IP address:

\begin{lstlisting}[language=json]
HOST_IP=192.168.120.16
ENABLED_SERVICES=n-cpu,rabbit,neutron,q-agt,q-l3
\end{lstlisting}

During the implementation and testing phase the Connectivity Manager Agent was installed on the controller node while the EMM was executed from the local machine.

\section{Connectivity Manager - Components and Operations}

\subsubsection{Selection of Best-Performing Hypervisor}


\textbf{Step 1: Retrieve current host utilization from CM Agent}

In order to decide on the placement the Connectivity Manager first of all needs to retrieve the host information from the Agent. It then sums up the different resource utilizations: amount of servers currently running on it, the total amount of used RAM \& vCPUs.
\begin{figure}[H]
\centering

\includegraphics[width=0.4\textwidth]{images/implementation/cm_get_host_utilization}

\caption{Check resource utilization of hosts}
\end{figure}

The amount of resources that are required in total to deploy the topology on the tenant needs to be calculated by adding up the amount of resources that are needed for each Unit. This can easily be done by checking its flavor.

\begin{figure}[H]
\centering

\includegraphics[width=0.5\textwidth]{images/implementation/cm_get_topology_requirements}

\caption{Get total amount of required resources for topology}
\end{figure}

\textbf{Step 2: Check if Topology is within the limitations of the Quota and currently available resources on the tenants hosts.}

In the second step it is checked if the topology is feasible for deployment. This decision is made by first comparing the values from Step 1 to the available Quota values (amount of VMs, RAM, vCPUs required for the sum of all servers) and secondly finding out if the required resources are less or equal to the currently not utilized resources in the tenants infrastructure.

\begin{figure}[H]
\centering

\includegraphics[width=0.6\textwidth]{images/implementation/cm_feasibility_check}

\caption{Deployment feasibility check}
\end{figure}

These comparisons need to be made, in order to conform with the requirement of not overcommitting any resources.


\textbf{Step 3: Check whether the Topology can be deployed on a single host.}

The next step checks if the total amount topology resources from Step 1 can be deployed on a single host. If there are multiple hosts that have enough capacity available, the first one within the list is returned by this method.

\begin{figure}[H]
\centering

\includegraphics[width=0.7\textwidth]{images/implementation/cm_single_host_check}

\caption{Single host deployment check}
\end{figure}


\textbf{Step 4: Set selected host as availability zone for each Unit.}

Lastly the chosen host needs to be set in the topology, so it can be returned to Heat for continuing with the deployment process. The syntax for a Units availability zone needs the 'nova' suffix, so this string needs to be added in front of the host name.

\begin{figure}[H]
\centering

\includegraphics[width=0.5\textwidth]{images/implementation/cm_set_az_topology}

\caption{Set AZ per Unit}
\end{figure}

\subsubsection{Enabling QoS for servers}

The rates for the QoS classes can be set in the configuration of the EMM.
The default location is \textit{/etc/nubomedia/emm.properties} and it contains the following default rates:
\begin{lstlisting}[language=commands]
# CONNECTIVITY MANAGER PROPS & QOS RATES (IN BIT/S)
cm_agent_ip=192.168.41.45
# GOLD = 100MBIT/S - 10GBIT/S
gold_min=100000000
gold_max=10000000000
# WHOLESALE = 100MBIT/S - 1GBIT/S
wholesale_min=100000000
wholesale_max=1000000000
\end{lstlisting}

For setting QoS for all Units within a Service Instance the following workflow is passed through:

\begin{figure}[H]
\centering

\includegraphics[width=0.6\textwidth]{images/implementation/cm_set_qos.png}

\caption{Method for setting QoS for all Units}
\end{figure}

The dictionary that this method returns is then converted to the JSON format and sent to the Connectivity Manager Agent by performing a HTTP POST to the \textit{/qoses} path of the IP address that is set for the \textit{cm\_agent\_ip} property in the configuration file.

\section{Connectivity Manager Agent - Components and Operations}

The two major operations that the Connectivity Manager needs to perform in order to provide the required services to this project is retrieving the resource status of the data center and setting QoS for the servers ports.

For using the OVS Client it needs to access the OVSDB that is located on each host. In order to allow that the following command needs to be executed on them once:
\begin{lstlisting}[language=commands]
$ sudo ovs-vsctl set-manager ptcp:6640
\end{lstlisting}

This enables remote administration through a passive TCP connection using the local IP address and port 6640. 

\subsection{Get list of all hosts and their utilization state}

The following figure displays the way in which the CM Agent creates a dictionary containing the list of all hosts within the selected tenant. It contains information such as the amount of already running virtual machines, the amount of allocated RAM in MB and the number of vCPUs in use. For each of the virtual machines it shows their ID, name and further resource information that is later needed for setting QoS. In case the VM already has a QoS assurance attached to its port, the corresponding minimal and maximal rates are shown.

\begin{figure}[H]
\centering

\includegraphics[width=0.65\textwidth]{images/implementation/cma_host_list}

\caption{Get list of hosts}
\end{figure}

In order to identify the rates that might have already been attached to a VMs port, a few steps need to be taken. With the servers IP address the Neutron Port ID can be retrieved. This Neutron Port ID has a Port ID within Open vSwitch that can be retrieved from the OVSDB. 

\begin{figure}[H]
\centering

\includegraphics[width=0.5\textwidth]{images/implementation/cma_get_ovs_port_server}

\caption{Get OVS Port ID for server}
\end{figure}

Each Port contains a field for QoS, which holds its QoS ID. This ID can then be filtered from the Queue table and the associated rates are listed in there.

\begin{figure}[H]
\centering

\includegraphics[width=0.7\textwidth]{images/implementation/cma_get_qos_id_for_ovs_port}

\caption{Get QoS ID for OVS Port}
\end{figure}

\subsection{Set QoS rates for all servers}

In order to set the bandwidth rates for the servers this method first retrieves the list of all hosts that was mentioned in the previous figure and description. When the \textit{Set QoS} method is called through a request to the API the method body from the HTTP POST is passed onto it. This dictionary contains all QoS rates that need to be set for each server. 


\begin{figure}[H]
\centering

\includegraphics[width=0.5\textwidth]{images/implementation/cma_set_qos}

\caption{Set requested QoS rates for all servers}
\end{figure}

The method that is called from within the above figure for setting QoS for a single server performs the following activities:

\begin{figure}[H]
\centering

\includegraphics[width=0.4\textwidth]{images/implementation/cma_set_qos_single_server}

\caption{Set QoS rates for a single server}
\end{figure}

\subsection{API}

The API was implemented using Bottle. It is a fast, simple and lightweight WSGI micro web-framework \cite{bottle-docs}. Therein three routes were defined:
\begin{lstlisting}[language=json]
# Welcome Screen
self._app.route('/', method="GET", callback=self._welcome)
# Host method
self._app.route('/hosts', method="GET", callback=self._hosts_list)
# QoS method
self._app.route('/qoses', method=["POST", "OPTIONS"], callback=self._qoses_set)
\end{lstlisting}

The first route contains a welcome message and can be used to check if the WSGI app is currently running.
The /hosts route calls the list\_hosts() method in the Agent when a HTTP GET is received.
When the /qoses route is called with the QoS parameters in its HTTP body it calls the set\_qos() method in the Agent class.

The server uses the localhost IP address and is served and listens on port 8091. It is important that this IP address is whitelisted in case a firewall exists. In the case of using a Vagrant box the port needed to be forwarded to the local machine.

\section{Tests}

For testing the components of the Connectivity Manager Agent the code was synchronized to a local testbed with two virtual machines, one of which acted as the control node and the other just as a compute node. 

\begin{figure}[H]
\centering

\includegraphics[width=0.17\textwidth]{images/implementation/cma_tests.png}

\caption{Connectivity Manager Agent: Test package}
\end{figure}

These functional tests contain the following methods of the main components:
\begin{itemize}
\item Agent: instantiates the Agent class and tests listing the information about all hosts
\item Keystone: tests retrieving the token and endpoint
\item Neutron: lists all ports and shows the Neutron Port ID for a selected IP address
\item Nova: retrieves all hosts and their servers
\item OVS: tests the main functionalities that are used within this work: listing all interfaces, ports, queues, qos's and creating a new queue and qos that are then attached to a OVS port
\item QoS: takes the parameters set in the \textit{test\_qos\_config.json} file as an input and posts it to the API in order to try and set QoS on the defined server ports
\item ReST: tests retrieving the list of all hosts through the API
\end{itemize}

The tests for the Connectivity Manager are part of the testing packages of the Elastic Media Manager, because that is the only way they're called. The \textit{test\_deploy.py} was used for checking if the integration works successfully.  It uses a predefined topology from a local file. The following minimized configuration was used during the tests:
\begin{lstlisting}[language=json]
{
    "name":"local_nm_template_minified",
    "service_instances": [
        {
            "name":"Controller",
            "service_type":"Controller"
        },
        {
            "name":"Broker",
            "service_type":"Broker"
        }
    ]
}
\end{lstlisting}

To find out more about the parameters that each of the service\_types corresponds to, please check the \textit{Topology Definition} % % LINK TO SECTION % %
in the Evaluation section.
\chapter{Evaluation}

\section{Feature analysis}



\section{Connectivity Manager integration with NUBOMEDIA project}

% % CM deliverable % %

The Connectivity Manager (CM) is part of the NUBOMEDIA platform and is placed between the virtual network resource management of the cloud infrastructure and the
multimedia application. The main focus of the CM is related to management and control of network functions of the virtual network infrastructure provided by OpenStack.

Nubomedia is an elastic Platform as a Service (PaaS) cloud for interactive social multimedia. Its architecture is based on media pipelines: chains of elements providing media capabilities such as encryption, transcoding, augmented reality or video content analysis. These chains allow building arbitrarily complex media processing for applications. As a unique feature, from the point of view of the pipelines, the NUBOMEDIA cloud infrastructure behaves as a single virtual super-computer encompassing all the available resources of the underlying physical network.


% % from EMM % %


Deployment
The deployment phase starts when an administrator requests the instantiation of
NUBOMEDIA. Inside the appropriate request, the administrator specifies what are the
capabilities required for this instance of the platform. Based on the information
received, the EMM needs to instantiate virtual resources on the Virtualized
Infrastructure. For this step there are several ways of doing it:
-Request directly OpenStack virtual networks and compute resources
-Request to Heat the required resources creating a template
For simplifying and making homogenous the whole process, we selected the second
option. The EMM has to create a Heat template based on the required resources which
are requested by the admin. Once the template is created, it is sent to Heat using its API.
At the end of the deployment process, Heat sends back all the information related with
the instantiated resources which the EMM can use for moving to the next phases.


Runtime
Once all the NUBOMEDIA components are deployed and started, it is needed to
actively check which specific levels of SLAs are met. The EMM provides a runtime
system which actively controls the situation of a group of NUBOMEDIA components,
and based on some policies, which the administrator inserted, decides whether to scale
in or out them.
When the administrator requests the instantiation of a NUBOMEDIA platform, it puts
also specific policies to specific elements. A policy is just a set of alarms and actions.
There are two different ways of realizing such system via triggering mechanisms:
- Actively: the EMM actively checks whether the conditions of the alarm are met.
For doing it, it typically requests to the monitoring system last values of the
metrics involved, and performs some basic operations for evaluating the status.
- Passively: the EMM pushes the alarm into the monitoring system. When the
conditions are met, the monitoring system triggers the alarm to the EMM.
For NUBOMEDIA Release 3, it was decided to use the first approach. Once the EMM
realizes that an alarm is in ACTIVE state, it has to trigger the execution of the action
contained in the Policy. Typically this policy can involve the instantiation/removal of
the NUBOMEDIA elements.
\section{Conclusion}
 \cleardoublepage
\chapter{Conclusion}

\section{Summary}

With the use of SDN and OpenFlow a framework for creating programmable networks already exist. However the integration and use of its full extent within OpenStack is not given today. The bandwidth supplied to different services can not be influenced and topologies that are deployed using the Orchestration tool Heat can not be placed on specific hosts. With this work the first step for those main objectives is made. With the help of the Open vSwitch configuration tool it is possible to get information about the current state of the switched network and enable QoS using Queues. The needed configuration and information retrieval can now be accessed through a ReST API. The Elastic Media Manager is able to access this information and make decisions about where the topology has the best-performing network connectivity. The Service-Level-Agreements can be easily changed and extended. The evaluation shows step-by-step the improvements in bandwidth rates with the use of the Connectivity Manager.

\section{Problems Encountered}

One big problem during the development process was getting a stable testbed running with the correct configuration. Devstack was a big help in this process, because not many manual corrections need to be made, other than adjusting the configuration file. However some of the features that are in the 'stable' version of OpenStack Juno still need further bug-fixing and can't be used in a production environment. During the test of the state-of-the-art solution a lot of time was spent on testing different versions, which was unsuccessful. Due to this fact, comparisons to other solutions for enabling Quality of Service can't be made. 
% % OVS JSON invalid? % %

\section{Future Work}

For future development the algorithm for selecting the best-performing host could take more dynamic factors into account. One possibility is to check the network-speed of all the host's network interfaces. It would be beneficial to implement QoS as a Neutron extension and make use of the already existing Neutron API. 

Integrate with Neutron API as Extension / Plugin
QoS Flow matching




\protect \addtocontents{toc}{\protect\newpage}  % Seitenumbruch im Inhaltsverzeichnis
\cleardoublepage


\begin{appendix}
\chapter{List of source codes}
\index{a2ps}\index{Pretty Printer}\index{Quelltexte}

\textcolor{darkred}{Anmerkungen: Quelltextausz�ge zu einer Implementierung sind im Anhang dann sinnvoll, wenn einige, spezielle Implementierungstechniken aufgezeigt werden sollen, die in der Darstellung als Algorithmus oder Pseudocode nicht deutlich werden. Keinesfalls soll der gesamte Quelltext angeh�ngt werden und weiterhin soll auch in einer Vorbemerkung die Auswahl der Quelltextausz�ge genau erkl�rt werden.}

\textcolor{darkred}{Verwendet wird das freie Quelltext-Pretty-Printing-Tool a2ps.exe mit folgender Aufrufkonvention (vgl. \mbox{[a2ps 07, grep 07]):}}


\verb$     a2ps.exe --pretty-print=cxx -i test.cpp -o test.ps -T3$

\medskip
\medskip

\textcolor{darkred}{Die Quelltextdateien d�rfen hierf�r eine Zeilenl�nge von 80 nicht �berschreiten. Die st�renden Kommentare im Header und Footer des entstandenen .ps-Files k�nnen mithilfe des freien Tools grep.exe automatisiert entfernt werden. Eine Batch-Datei f�r den gesamten Konvertierungsprozess inklusive Konvertierung in das pdf-Format hat beispielsweise folgenden Inhalt:}


\verb$     a2ps --pretty-print=cxx -i %1 -o tmp -T3$\\
\verb$     grep -v "Gedruckt von" tmp | grep -v ") footer" > %1.ps$\\
\verb$     del tmp$\\
\verb$     "c:\programme\adobe\acrobat 7.0\Acrobat\acrobat.exe" %1.ps$\\


                 % C
\chapter{Glossar}

\interlinepenalty=10000 % keine Schusterjungen, keine Hurenkinder



\begin{description}

\item[\bf{SDN}] $\rightarrow$ Software-Defined Networking.

\end{description}
\interlinepenalty=100



                    % E
\end{appendix}



% Erstes Literaturverzeichnis, ohne BibTeX

\interlinepenalty=10000 % Literatureintr�ge: Abs�tze zusammenhalten
\cleardoublepage
\addcontentsline{toc}{chapter}{Literatur}   
%F�r das nachfolgende exemplarische Literaturverzeichnis wurde die einfache thebibliography-Umgebung von Latex verwendet. F�r Studien- und Diplomarbeiten mit weniger als 40 Quellen sollte diese auf jeden Fall ausreichen und hat gegen�ber dem komplexen Bibtex-Paket weiterhin den Vorteil flexiblerer Formatierungsm�glichkeiten.
%
%Im Anschluss an das exemplarische Literaturverzeichnis ist ein zweites Verzeichnis beigef�gt, welches weiterf�hrende Quellen zum vorliegenden Latex-Template enth�lt: Download-Links zur Software, freie Online-Latex-Handb�cher usw.



\begin{thebibliography}{Ti}



\bibitem[1]{roadtosdn} N. Feamster, J. Rexford, E. Zegura: \glqq The Road to SDN\grqq, ACM Queue, Volume 11, Issue 12, December 30, 2013.
\bibitem[2]{onfnewnorm} Open Networking Foundation: \glqq Software-Defined Networking: The New Norm For Networks \grqq, ONF White Paper, April 13, 2012.
\bibitem[3]{onfdefinition} Open Networking Foundation: \glqq Software-Defined Networking (SDN) Definition \grqq, \url{https://www.opennetworking.org/sdn-resources/sdn-definition}
\bibitem[4]{ofversion13} S. Natarajan, SDN Hub: \glqq OpenFlow version 1.3 tutorial \grqq, \url{http://sdnhub.org/tutorials/openflow-1-3/}
\bibitem[5]{ofspecification} Open Networking Foundation: \glqq OpenFlow Switch Specification \grqq, Version 1.3.0, June 25, 2012.
\bibitem[6]{ovs-faq} Open vSwitch: \glqq Open vSwitch - Frequently Asked Questions \grqq, \url{http://git.openvswitch.org/cgi-bin/gitweb.cgi?p=openvswitch;a=blob_plain;f=FAQ;hb=HEAD}
\bibitem[7]{ovs-readme} Open vSwitch: \glqq Open vSwitch - Readme \grqq, \url{https://github.com/openvswitch/ovs/blob/master/README.md}
\bibitem[8]{prognetworkingovs} J. Gross, VMware: \glqq Programmable Networking with Open vSwitch \grqq, LinuxCon, September, 2013. \url{https://events.linuxfoundation.org/sites/events/files/slides/OVS-LinuxCon\%202013.pdf}
\bibitem[9]{ovsdeepdive} J. Pettit, E. Lopez, VMware: \glqq OpenStack: OVS Deep Dive \grqq, 07 November, 2013.
\bibitem[10]{ovsdbmanual} Open vSwitch: \glqq Open vSwitch Manual\grqq, \url{http://openvswitch.org/ovs-vswitchd.conf.db.5.pdf}
\bibitem[11]{tc-manual} B. Hubert: \glqq tc - Linux man page\grqq, \url{http://linux.die.net/man/8/tc}
\bibitem[12]{htb-guide} M. Devera, D. Cohen: \glqq HTB Linux queuing discipline manual\grqq, \url{http://luxik.cdi.cz/~devik/qos/htb/manual/userg.htm}
\bibitem[13]{htb-qdiscs} J. Vehent: \glqq Journey to the Center of the Linux Kernel: Traffic Control, Shaping and QoS\grqq, \url{http://wiki.linuxwall.info/doku.php/en:ressources:dossiers:networking:traffic_control}
\bibitem[14]{gre-rdo} RDO: \glqq Using GRE tenant networks
\grqq, 2014, \url{https://openstack.redhat.com/Using_GRE_Tenant_Networks}
\bibitem[15]{openstack-ops} OpenStack Foundation: \glqq OpenStack Operations Guide\grqq, February 01, 2015, \url{http://docs.openstack.org/openstack-ops/openstack-ops-manual.pdf}
\bibitem[16]{openstack-ops} OpenStack Foundation: \glqq OpenStack Configuration Reference\grqq, Compute-Scheduler section, February 01, 2015, \url{http://docs.openstack.org/juno/config-reference/content/section_compute-scheduler.html}
\bibitem[17]{openstack-compute} OpenStack Foundation: \glqq OpenStack Compute\grqq,  2015, \url{https://www.openstack.org/software/openstack-compute/}
\bibitem[18]{openstack-installjuno} OpenStack Foundation: \glqq OpenStack Installation Guide\grqq, January 29, 2015, \url{http://docs.openstack.org/juno/install-guide/install/apt/content/}
\bibitem[19]{openstack-training} OpenStack Foundation: \glqq OpenStack Training Guide\grqq, Chapter 7, January 31, 2015, \url{http://docs.openstack.org/training-guides/content/operator-network-node.html}
\bibitem[20]{odl-intro} K. Mestery, D. Meyer, Linux Foundation: \glqq Introduction to OpenDaylight and Hydrogen\grqq, 2014, \url{https://www.openstack.org/assets/presentation-media/osodlatl.pdf}
\bibitem[21]{odl-ovsdb} C. Dixon, OpenDaylight: \glqq OVSDB Integration\grqq, October 1, 2014, \url{https://github.com/opendaylight/docs/blob/master/manuals/developers-guide/src/main/asciidoc/ovsdb.adoc}
\bibitem[22]{odl-ovsdb} D. Pemberton, A. Linton, S. Russell, University of Oregon: \glqq RYU OpenFlow Controller\grqq, 2014, \url{https://nsrc.org/workshops/2014/nznog-sdn/raw-attachment/wiki/WikiStart/Ryu.pdf}
\bibitem[23]{ryu-start} D. Pemberton, A. Linton, S. Russell, University of Oregon: \glqq RYU OpenFlow Controller\grqq, 2014, \url{https://nsrc.org/workshops/2014/nznog-sdn/raw-attachment/wiki/WikiStart/Ryu.pdf}
\bibitem[24]{ryu-comparison} Yamamoto, OpenStack: \glqq Neutron OFAgent - Comparison with OVS\grqq, December 5, 2014, \url{https://wiki.openstack.org/w/index.php?title=Neutron/OFAgent/ComparisonWithOVS&oldid=69704}
\bibitem[25]{neutron-qos} S. Collins, OpenStack: \glqq Neutron QoS\grqq, October 25, 2013, \url{https://wiki.openstack.org/w/index.php?title=Neutron/QoS&oldid=34054}
\bibitem[26]{openstack-admin} OpenStack Foundation: \glqq OpenStack Admin User Guide\grqq, February 01, 2015, \url{http://docs.openstack.org/user-guide-admin/user-guide-admin.pdf}
\bibitem[27]{nubomedia} L. Lop�z, NUBOMEDIA: \glqq What's NUBOMEDIA?\grqq, \url{http://nubomedia.eu/page/whats-nubomedia}


\end{thebibliography}






% Zweites Literaturverzeichnis, mit BibTeX

%\renewcommand\bibname{Weiterf�hrende Literatur zu wissenschaftlichen Ausarbeitungen}
%\nocite{*} % auch die nicht verwendeten bibtex-Eintr�ge einblenden
%\cleardoublepage
%\addcontentsline{toc}{chapter}{Weiterf�hrende Literatur zu wissenschaftlichen Ausarbeitungen}
%\bibliography{bibliografie}
%\interlinepenalty=100



% Sachverzeichnis einf�gen

\renewcommand\indexname{Sachverzeichnis}
\cleardoublepage
\addcontentsline{toc}{chapter}{Sachverzeichnis}
\linespread{0.99} % Abhilfe zu Schusterjungen .. im Index. Die Zahl ist entspr zu variieren
\printindex                                 



% Schmutzblatt (leere Seite am Ende)

\newpage                                    
\pagestyle{empty}
\begin{figure}[H]
\centering
\includegraphics[width=0.9\textwidth]{images/general/leer.jpg}
\end{figure}



\end{document}




















%
